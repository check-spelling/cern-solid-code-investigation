\begin{abstract}
    Tim Berners-Lee invented the Web to distribute information freely at CERN. The undeniable growth and dependence have created a data exploitation dilemma. Solid is a set of technical reports, an ecosystem, and a movement to regain data sovereignty and empower the users of the Web. CERN defined a project to investigate Solid and how CERN can be involved. The work presented in this paper is tackling part of this project, where the Solid principles will be applied to software at CERN in an experimental approach. Relevant software at CERN and in Solid had been identified and analyzed. Two Solid apps were developed, evaluated, and analyzed. The work concluded that Solid development is challenging in fields of \textit{Performance} and \textit{Usability}. When every user stores their data, it is difficult for traditional applications to retrieve the data from hundreds of users promptly. Flows of granting access to sensitive information are also still complex and confusing. Solid is maturing but still requires a lot of work and expertise from many professions. CERN is in an ideal situation to extend its collaboration by integrating the decentralized storage into their infrastructure and migrating or creating new applications with Solid principles in mind.
\end{abstract}