\section{Proceedings in the CERN-Solid Collaboration}

With the two prototypes developed, integrated into Indico, and deployed to a running instance and showing a working solution with Solid principles in Indico, how can \gls{cern} proceed with its collaboration attempt? In the following the two fields of Solid specification implementations and Solid apps shall be looked at and debated how \gls{cern}'s future with Solid can look like.

\subsection{Servers}

The in \cite{cern-solid-investigation-spec} identified Solid implementations remain where they were at the time of composing the research paper \cite{cern-solid-investigation-spec}. A lot of development has happened for the \gls{css}, but it is still in a minor release version and therefore not yet deployed and switch out with \gls{nss} on the public solidcommunity.net domain. This is expected to happen as soon the developers trust the state of \gls{css} to facilitate at least the functionality of \gls{nss} and the tooling is in place to transfer all existing data pods currently hosted on the web server. No publicly communicated date has been set in stone for this to take place, but can be assumed to happen this year or next. The Solid specification on the other hand have a clearly defined completion date of 30.06.21 \cite{solid-tr}. Once the technical reports are completed -- but of course remain in the state of a living standard -- it can be expected to ease development process in the Solid ecosystem, as no major changes in the specification mean no critical new features need to be developed or supported by the server developments.

The \gls{nss} is maintained by a small team of unfunded open-source developers and only has capacity to fix major bugs and keep the dependencies up-to-date. It is still the most used Solid implementation and recommended data pod solution in conjunction with one of the providers, namely solidcommunity.net \cite{solid-community}.





\subsection{Applications}
