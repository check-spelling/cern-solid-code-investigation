
% \setcounter{section}{0}

% \begin{savequote}[10cm]
% Data is a precious thing and will last longer than the systems themselves.
% \qauthor{Tim Berners-Lee}
% \end{savequote}

\chapter{Conclusion}

Decentralizing a centralized system is complex, and even developing and integrating small-scale Solid modules that harmonize with sophisticated software is. In this paper shown work has shown in an experimental approach how the Solid ecosystem can enrich \gls{cern} software, its challenges and has laid out a possible future for the collaboration of the two.

The paper presented two working Solid apps embedded into Indico and has shown how data can reside in user's control while still be served purposeful to others. The applications have also shown how existing information in one's data pod can be used in other systems. These two insights from storing data outside of Indico and enabling information to flow back into the system has shown admirable features for the future of the Web: control through external data pods and reuse thanks to interoperable data formats of Linked Data. The third principle of Solid, decentralized authentication was also enabled and necessary for the development, but had a lesser impact due to restricting the prototype to only Indico and not guiding testing participants to other applications. Further, a decentralized \gls{sso} service is already in use at \gls{cern}.

The implementation of Solid comes with imperfections. The detailed analysis and evaluation of the Solid apps have shown those flaws in the synergy between Indico and Solid principles. The flaws being mainly performance regressions in the retrieval and distribution of data from several external data pods. Where in centralized systems the data is stored in one place and can be processed and prepared for clients neatly, in a decentralized architecture several requests are required to retrieve the data first. The prototypes' design was as decoupled as possible from Indico to not infiltrate Indico's functionality and allow other systems to integrate the modules as well. The decoupling has proven that the two systems are too distant and implementations need to be closer to Indico. A more imminent solution means Indico is required to change too. The data flow from the data pod back in to Indico, as demonstrated in the second implementation of the \gls{poc}, is challenging by reason of lacking any semantic structure whatsoever in Indico.

\vspace{1cm}

Solid as an ecosystem



\vspace{3cm}

* “you will get back to the problem definition and summarize what you have done”

* plug and play does not work so good, closer to Indico -> Indico proxy

* Solid lives through the community, when only one company does the implementation it will not thrive
* vicious cycle
* 

Solid from the community, for everyone
start by the community
get going, but breakthrough takes more -> CERN

community <-> company
not just one company it won't thrive

software needs to be rethought
major challenges:
    usability
    performance
    describing the world in Linked Data, can the data of the complex world be described for by machines?
it takes entities as CERN to not let commercial companies built the products, otherwise Solids futures might be at stake? (a bit harsh)

solid is still experimental and needs a lot of attention from developers
startups are good
the more the merrier
the community needs to remain as a driver
high profile people help, but cannot be the motivator

build better and hybrid products for people where they might not even realize the decentralized benefits

challenge the known