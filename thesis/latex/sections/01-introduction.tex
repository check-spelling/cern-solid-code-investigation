\chapter{Introduction}
% \setcounter{section}{0}
\section{Context}

The Web was created in 1989 by Sir Tim Berners-Lee while working at the institution of \gls{cern} “\textelp{} to allow people to work together by combining their knowledge in a web of hypertext documents” \cite{timbl-bio}. This brilliant idea has ever since grown as an essential part of our lives \cite{cern-solid-investigation-spec}. While bringing a new platform for innovation into existence, a new level of oppression and surveillance that has never been seen before has facilitated without most even realizing it. Data are harvested and analyzed to generate models describing and predicting human behavior far beyond what is morally tolerable. These actions permit large amounts of wealth as capitalistic and governmental bodies enjoy a great interest in these models to generate more profit or more control. The escalation of attraction towards data has led to the construction of so-called \textit{data silos}. Data gatherers built attractive applications for users while locking them into their walled gardens to have all the data for themselves and then sell it.

Through whistle-blowing, active journalism, and technical education, a new wave of hope arises in the ocean of data hunting. Not giving up on the Web, which has in its little over 30 years of lifetime proven its paramount act in state, society, and the economy, Sir Tim has an adjustment for the Web specified. With the help of several Web enthusiasts, companies, and even governments, they founded the idea of Solid.

Solid aims to return data control to the users. To achieve full data sovereignty, it specifies a new way of building applications on the Web through several technical reports. That lets the users control their data and can decide who can see what. Like the Web, which was also defined by technical reports and prototypes, it is for everyone and is not limited by anything.

“Being the Web’s birthplace, \gls{cern} remains a High Energy Physics laboratory; hence, its primary mission is to run an accelerator, its detectors, and the relevant experiments. Computing is of paramount importance for filtering, storing, distributing, accessing, analyzing the experimental data. Nevertheless, due to its large and distributed user base, \gls{cern} offers sophisticated solutions on all software application fronts. In terms of price and transparency, proprietary packages have been disappointing. Following the raising worldwide awareness of personal data ownership and sovereignty, \gls{cern} is interested in Solid.” \cite{cern-solid-investigation-spec}

\section{Goal}

The presented paper 

The work presented in this paper will conclude the \textit{\gls{cern}-Solid Code Investigation} project \cite{cern-solid-code-investigation-project-description}. Where the previous job done in \cite{cern-solid-investigation-spec} focused on analyzing the Solid ecosystem, this body of work will focus on the remaining points defined in the project description \cite{cern-solid-code-investigation-project-description}. The remaining issues include developing two modules integrating Solid principles programmatically into existing software from \gls{cern}, these will be presented with their design choices, and the challenges met. Further, these developments will be evaluated and analyzed based on the quality attributes of security, performance, and usability. Followed by a section identifying the challenges with Solid in its current state and how traditional software, namely one from \gls{cern} called Indico, has issues adopting a \textit{Solid-way} of building software. The last section will introduce a possible continuation for \gls{cern} as this project aims to determine how \gls{cern} can be involved in the Solid ecosystem and the fate of the Web.
