\chapter{Introduction}
% \setcounter{section}{0}
\section{Context}

The Web was created in 1989 by Sir Tim Berners-Lee while working at the institution of \gls{cern} “\textelp{} to allow people to work together by combining their knowledge in a web of hypertext documents” \cite{timbl-bio}. This brilliant idea has ever since grown as an essential part of our lives \cite{cern-solid-investigation-spec}. While bringing a new platform for innovation into existence, a new level of oppression and surveillance that has never been seen before has facilitated without most even realizing it. Data are harvested and analyzed to generate models describing and predicting human behavior far beyond what is morally tolerable. These actions permit large amounts of wealth as capitalistic and governmental bodies enjoy a great interest in these models to generate more profit or more control. The escalation of attraction towards data has led to the construction of so-called \textit{data silos}. Data gatherers built attractive applications for users while locking them into their walled gardens to have all the data for themselves and then sell it.

Through whistle-blowing, active journalism, and technical education, a new wave of hope arises in the ocean of data hunting. Not giving up on the Web, which has in its little over 30 years of lifetime proven its paramount act in state, society, and the economy, Sir Tim has an adjustment for the Web specified. With the help of several Web enthusiasts, companies, and even governments, they founded the idea of Solid.

Solid aims to return data control to the users. To achieve full data sovereignty, it specifies a new way of building applications on the Web through several technical reports. That lets the users control their data and can decide who can see what. Like the Web, which was also defined by technical reports and prototypes, it is for everyone and is not limited by anything.

“Being the Web’s birthplace, \gls{cern} remains a High Energy Physics laboratory; hence, its primary mission is to run an accelerator, its detectors, and the relevant experiments. Computing is of paramount importance for filtering, storing, distributing, accessing, analyzing the experimental data. Nevertheless, due to its large and distributed user base, \gls{cern} offers sophisticated solutions on all software application fronts. In terms of price and transparency, proprietary packages have been disappointing. Following the raising worldwide awareness of personal data ownership and sovereignty, \gls{cern} is interested in Solid.” \cite{cern-solid-investigation-spec}

\section{Goal}

This paper tackles a portion of a project defined at \gls{cern}. \textbf{The project in its entirety investigates how two architectural contrasting systems can be combined or how a decentralized software ecosystem can integrate into a centralized one}. The two systems are Indico, which acts in this context as the traditional and centralized software monolith, and Solid, the decentralized ecosystem, which decouples authentication, application, and data from each other. The first part of this \textit{\gls{cern}-Solid Code Investigation} project \cite{cern-solid-code-investigation-project-description} has been done in a previous \textit{research project} \cite{cern-solid-investigation-spec}, where the two entities \gls{cern} and Solid were introduced, and an overview of both ecosystems was given. Further, the paper identified software at \gls{cern} and generally analyzed it based on their quality and compatibility for integration with Solid. The paper also looked at the Solid ecosystem and how the concept of data sovereignty is technically translated.

The following steps in the project are presented in this body of work. In an experimental approach, two Solid modules are developed and embedded into \gls{cern}'s Indico system. The adopted design of the modules is critically put into perspective with the many possibilities met, and its architecture is thoroughly evaluated. Challenges, advantages, and gaps in programming and integration, and existing solutions are described and documented. A possible continuation of how \gls{cern} can be involved in the Solid ecosystem and the fate of the Web is debated and a recommendation put in place.
