\chapter{Introduction}
% \setcounter{section}{0}
\section{Context}

The Web was created in 1989 by Sir Tim Berners-Lee while working at the institution of \gls{cern} “\textelp{} to allow people to work together by combining their knowledge in a web of hypertext documents” \cite{timbl-bio}. This brilliant idea has ever since grown as an essential part of our all lives \cite{cern-solid-investigation-spec}. While bringing a new platform for innovation into existence, a new level of oppression and surveillance that has never been seen before has facilitated without most even realizing. Data are harvested and analyzed to generate models describing and predicting human behavior far beyond what is morally tolerable. These actions permit the generation of large amounts of wealth as capitalistic and governmental bodies enjoy a great interest in these models to either generate more profit or more control. The escalation of attraction towards data has led to the construction of so called \textit{data silos}, where data gatherers built attractive applications for users while locking them into their walled gardens to have the all data for themselves and then selling it.

Through whistle blowing, active journalism, and technical education a new wave of hope is arising in the ocean of data hunting. Not giving up on the Web, which has in its little over 30 years of lifetime proven its paramount act in state, society, and the economy, Sir Tim has an adjustment for the Web specified. With the help of several Web enthusiasts, companies, and even governments the idea of Solid was found.

Solid aims at giving the users back the control over their data to regain full data sovereignty. It does so by specifying through a number of technical reports a new way of building application in the Web that let the users control their data and in this way can decide who can see what. Just like the Web, which was also defined by technical reports and prototypes, it is for everyone and not limited by anything. 

“Being the Web’s birthplace, \gls{cern} remains a High Energy Physics laboratory; hence, its primary mission is to run an accelerator, its detectors, and the relevant experiments. Computing is of paramount importance for filtering, storing, distributing, accessing, analyzing the experimental data. Nevertheless, due to its large and distributed user base, \gls{cern} offers sophisticated solutions on all software application fronts. In terms of price and transparency, proprietary packages have been disappointing. Following the raising worldwide awareness of personal data ownership and sovereignty, \gls{cern} is interested in Solid.” \cite{cern-solid-investigation-spec}

\section{Goal}

The work presented in this paper will conclude the \textit{\gls{CERN}-Solid Code Investigation} project. Where the previous work done in \cite{cern-solid-investigation-spec} focused on analyzing the Solid ecosystem, this body of work will focus on the remaining points from the project description in \cite{cern-solid-code-investigation-project-description}. The developed \glspl{poc}, that bridge \gls{cern} software with Solid ideas, will be presented and their design choices and the challenges met. The developments will be evaluated and analyzed based on the quality attributes, which had been defined with the relevant stakeholders.


* cite project descrip. here: this part of previous work where spec was analyzed