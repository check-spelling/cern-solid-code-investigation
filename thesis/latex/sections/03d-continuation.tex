\section{Continuation in the CERN-Solid Collaboration}

With the two prototypes developed, integrated into Indico, and deployed to a running instance and showing a working solution with Solid principles in Indico, how can \gls{cern} proceed with its collaboration attempt? In the following, the two fields of Solid specification implementations and Solid apps shall be looked at and debated what \gls{cern}'s role in the future of Solid can be.

\subsection{Solid Servers}

The in \cite{cern-solid-investigation-spec} identified Solid implementations remain where they were when composing the research paper \cite{cern-solid-investigation-spec}. A lot of development has happened for the \gls{css}, but it is still in a minor release version and therefore not yet deployed and switch out with \gls{nss} on the public solidcommunity.net domain. This is expected to happen as soon the developers trust the state of \gls{css} to facilitate at least the functionality of \gls{nss} and the tooling is in place to transfer all existing data pods currently hosted on the web server. No publicly communicated date has been set in stone for this to occur but it can be assumed to happen this year or next. The Solid specification, on the other hand, has a clearly defined completion date of 30.06.21 \cite{solid-tr}. Once the technical reports are completed -- but of course remain in the state of a living standard -- it can be expected to ease the development process in the Solid ecosystem, as no significant changes in the specification mean no critical new features need to be developed or supported by the server developments.

The \gls{nss} is maintained by a small team of unfunded open-source developers and can only fix critical bugs and keep the dependencies up-to-date. It is still the most used Solid implementation and recommended data pod solution in conjunction with one of the providers, namely solidcommunity.net \cite{solid-community}.

\gls{cern} has a number of options in its continuation with Solid servers in their current status.

\begin{enumerate}
    \item Outstanding solution through solidcommunity.net
    \item Integration with CERNBox
    \item Sandboxed \gls{css}
    \item Develop own server solution
\end{enumerate}

\paragraph{Outstanding Solution}\mbox{}\\

The usage and testing of the \glspl{poc} required a WebID and a data pod. Through extensive research and careful considerations, it was concluded and recommended to all \gls{poc} participants to obtain the necessities through solidcommunity.net. The Solid Community is currently the most attractive data pod provider through its physical hosting in the \gls{uk} and openness regarding data usage and usage of \gls{nss} the go-to open-source Solid server solution. \gls{cern}, by policy, opts for data storage in European locations based on \gls{oc11} \cite{oc11}] and \gls{gdpr} \cite{gdpr}, and that all data are merely used to provide its services \cite{policy-cern-server}. Storage locations outside Europe can be possible only if there is a clear justification officially approved by the \gls{cern} \gls{odp}. This is also a reason why a hosted instance through Inrupt \cite{inrupt} not acceptable. Not to mention the expected cost of such a service. An outstanding solution also gives away control over the running version of the Server implementation. As of now most data pod providers run the \gls{nss}, but might switch over to \gls{css} -- which is desirable but could bring new challenges. Even though a server implementation can only be called a Solid server when it adheres to the Solid specification, which can be tested by an \gls{ists} \cite{solid-test-suite}, which also means when the specification is followed, features should be equally supported and implementations should allow interoperability. This has been proven to be difficult when looking at the several browsers, which are all implementing the \gls{http}, \gls{html}, and \gls{uri} standards (and many more specifications) but are all behaving slightly differently. This might be due to missing resources to stay up to date with feature development, or contrasting interpretations of the specifications. These risks will always endure and hence bring challenges.
\vspace{0.5cm}
\paragraph{Integration With CERNBox}\mbox{}\\

\gls{cern} uses CERNBox \cite{cernbox} to provide personal cloud storage to all \gls{cern} users to host and share files. The service is based of ownCloud \cite{owncloud} and hosted on \gls{cern} premise. In 2016 Nextcloud \cite{nextcloud} was formed from a fork of the open-source core software of ownCloud. PDS Interop \cite{pds-interop}, a collective of open-source developers, has developed a Nextcloud \cite{nextcloud} plugin to make the file hosting service Solid compatible. A Nextcloud server with the Solid plugin enabled passes currently the complete \gls{ists}. The integration with a running Nextcloud instance is as easy as installing the plugin through the web interface of Nextcloud.

An integration with CERNBox seems to be a suitable option, provided resources are attributed for taking care of operational requirements and security preoccupations. Adding the Solid implementation into its own cloud storage infrastructure means the CERNBox administrators now also have to administer the Solid Nextcloud plugin. Performance issues due to the thousands of CERN users should be studied in advance. A possibility could be to only enable it for a subset of users to test the integration. Resources for these steps would have to be planned. Most importantly, the attack surface increased by enabling a plugin that includes an extra sharing functionality would have to be analyzed.
\vspace{0.5cm}
\paragraph{Sandboxed \gls{css}}\mbox{}\\

A less risky solution would be to use a sandboxed Solid implementation, such as \gls{css}. Once \gls{css} is released under a major version it could be deployed to self-contained OpenStack VM instances and then run as a new file storage system. This might seem redundant, considering \gls{cern} is already running a cloud storage system with CERNBox, but the added risk is much lower to infiltrate an existing system. The new \gls{css} deployments could solely be used for Solid related file storage and sharing and then incrementally be more integrated with existing \gls{cern} solutions. Even though the complexity might not be as high as with the installation through the Nextcloud plugin in CERNBox it still requires administrative work to keep the deployments running and updating \gls{css} regularly. Further, to even add more value and usability to this integration the \gls{css} could be used with the \gls{cas}. Meaning, instead of identifying with an external \gls{idp}, the \gls{cern} personnel could authenticate with their existing \gls{cern} accounts using \gls{cas}, which would need to be extended to also provide WebIDs.

Open-source and community-owned software brings a lot of advantage but also needs to be tread lightly. Software as this always relies on independent developers to fix bugs, develop features, which will not happen as fast when using a product with \glspl{sla}. A motivator for \gls{cern} to get involved with the development of open-source Solid services.
\vspace{0.5cm}
\paragraph{Own Server Solution}\mbox{}\\

Because Solid is an open standard and allows free development of own solutions it is always a possibility to develop a \gls{cern} Solid server from scratch. This is from all solutions the most ambitious one and least recommended, as it demands too many resources and a sophisticated solution through \gls{css} is almost release, which would benefit from additional given resources.

\subsection{Solid Applications}

Uncertainties in the adoption of a CERN Solid server do not help to advocate the usage of Solid applications or even the development of new Solid services. The \glspl{poc} have shown realistic use-cases for decentralized data storage. The architectural evaluation has shown the maturity, but also the flaws that would need to be tackled in order for them to be adopted into the production instance of Indico at \gls{cern} and then it also needs to be analyzed if those \glspl{poc} would even find users or it would be just a dead feature. Indico has also been established to not be the perfect candidate for a complete Solid overtake in its architecture. Therefore, less complex software at \gls{cern} might be a better target for further Solid-based developments. One prospect is the \gls{cern} Slides' App \cite{cern-slides}, which is a web application recently developed as part of a student project and is in production-ready quality and planned to be used as a main slides maker for \gls{cern} users. A possibility for \gls{cern} to remain present in the Solid ecosystem is to further the development of the Slides' App and even make it Solid compatible. The final resource combining all slides into a single presentation could be stored in a data pod, and even further thought all the information on the slides could be connected to resources from data pods and thus leveraging a decentralized approach to data management. Everything would be connected, could be reused, and would be completely interoperable. Information could still come from non-\gls{rdf} resources but could be transformed to \gls{rdf} upon entry. Resulting in an attractive hybrid application, that could be used without the need of a data pod or WebID, but would still find users as it is a wanted software at \gls{cern} and would bring the decentralized resource management to the ones who want to use it.

Another minor idea in the field of application development is based from the second \gls{poc} the autocomplete module, that the existing \gls{cern} application could be enriched by using more structured data concepts in their code. As an example attache the \gls{html} \texttt{autocomplete} attribute to known input fields. Add more common input fields to allow the pre-definition of semantic structure on them. This way the forms can be reasoned about easier by machines, but this effort should not be limited to \gls{html} forms and can be extended to all sorts of fields where Schema.org has a description for.

\subsection{Recommendation}

A recommendation based on the previous two talked about fields of server implementation and application development for \gls{cern} is as follows.

The current efforts from \gls{cern} are rather limited in resources, general interest is present, but many factors hinder further active involvement with Solid. A public communication channel used to share milestones in the investigation is left unread. Announcements and call for testings of the \glspl{poc} in \gls{cern} internal email groups reaching hundreds of \gls{cern} users are met with participation, compliments, and comments, but still small in number. A presentation given at the HEPiX \cite{hepix} was well received by international Information Technology staff, from High Energy Physics and Nuclear Physics laboratories and institutes. So, a general curiosity exists, and a few enthusiasts are awaiting a Solid-like project to be supported by \gls{cern}, but it seems more generally that \gls{cern} is interested in an observing position as of now. This was concluded from the support given through this project's lifetime and the most recent decision of not funding through hiring for a fellowship position to work on a continuation of the \gls{cern}-Solid collaboration project.

The most likely future scenario is to work in collaboration with students from universities to work on further Solid related projects. By doing this \gls{cern} can stay closely connected to the Solid ecosystem and receive the latest developments in server implementations, specification evolution, and new exciting Solid apps. Once the Solid ecosystem has matured enough and has shown through other Solid app developments the possibilities and innovations with Solid, \gls{cern} can reevaluate its position and consider participating more actively by either Solid app development or hosting Solid pods for its personnel.
