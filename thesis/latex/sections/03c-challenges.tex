\section{Challenges, Advantages, and Gaps of Existing Solid Solutions versus CERN Ones}

\subsection{Lack of Solid Applications}\label{challenges:lack}

The relatively young lifetime of Solid, the only recent matured specifications, the rather unorthodox way of developing applications, vague documentation on how to build for the ecosystem, and sheer lack of motivation to develop a solution for a small group of enthusiasts are significant reasons for the lack of sophisticated Solid applications so far. It is hard for non-developers to engage and help in an ecosystem so young. Of course, many experts are needed to shape Solid and data sovereignty, but it takes developers to build applications that everyone can use. Without such applications, Solid mainly remains theoretical or relies on developers to show the signs of progress and opportunities. The \textit{causality dilemma} can be blamed for this. It takes developers to build the software, but no one will build solutions if no one engages because of lacking solutions. But the future is looking bright as the specifications have an approximate date of completion for the summer of 2021.

\subsection{Encryption at Rest}

Just like \gls{http} is specified without an encryption layer, so is Solid. The boards of the multiple Solid panels \cite{solid-panels} decided that encryption in Solid is not a problem to be solved for now. \gls{http} brings encryption with \gls{https} and ensures that all communications over \gls{http} are encrypted in the Solid ecosystem. For encryption at rest, meaning at the place where the data is stored and retrieved, no encryption is intended. The standards around Solid describe how the access to information is contracted, but if this data is encrypted is up to the storage mechanism, which is up to the provider of the data pod \cite{solidproject-faqs}.

\gls{cern} has a high interest in data sovereignty and requires all their external services to store the collected data on \gls{eu} soil, where the \gls{gdpr} support the data authors. For the cases where it is impossible, \gls{cern} has its own data centers to store data. In a scenario where \gls{cern} seeks a pod provider to enable Solid for their users, and the provider cannot guarantee \gls{gdpr} compliance or encrypts data in the data center, it is unlikely for \gls{cern} to adopt such a solution.

A hosting on-premise solution also comes with challenges. The involvement with Solid changes from user to the maintainer, as a self-hosted solution, means it needs to be ensured services are operating as expected at all times. These tasks always demand many resources, especially with outside solutions, where an in-house engineer team cannot fix bugs, instead rely on other developers to fix them. Open-source solutions are especially of high risk, since they do not come with any \glspl{sla}.

\subsection{User Interface}

A Solid server is a file system-based web server. Modern computers ship with graphical \glspl{os} to offer a good user experience when interacting with the file system on such computers. A Solid server is a place for users to manage their data requires a good user experience and a graphical \gls{os}. The great benefit of using Linked Data to enable interoperability for Solid applications has led to many solutions for connecting and using a data pod. One system is called SolidOS \cite{solidos} and acts as the \gls{os} for Solid servers. SolidOS has many great features that allow the direct use of Linked Data on one's pod, but it comes with a challenging \gls{ui}. The \gls{ui} is powerful but filled with bugs and counter-intuitive interactions. For example, a user is often prompted to drag and drop a WebID \gls{uri} to add the agent as a friend instead of allowing a regular text input to either copy and paste or type the WebID \gls{uri}.

SolidOS is currently the best and most sophisticated solution by features; hence it comes shipped with \gls{nss} deployments on most pod providers. These \gls{ui} complications make it extremely strenuous even for professionals to use one's pod and even more ambitious to onboard newcomers.

\subsection{Commercial and Open-Source Solutions}

What started with Sir Tim Berners-Lee as an idea and initial documentation of the idea has evolved dramatically. Today, Solid is an open standard driven by the Solid community. Everybody who wants to participate, help, steer the direction of the movement is welcome and can join at any time. Web enthusiasts do the majority of work in their spare time. Sometimes foundations or governments reward the developer’s work, such as the \gls{eu}, which has funded several Solid-related projects. A significant number of startups have also spotted the potential for Solid and now hold large teams. The most notable is Sir Tim's co-founded company Inrupt \cite{inrupt}, which focuses entirely on Solid. These companies are invaluable and necessary for the success of Solid, but they also bring some complications. In recent progressions, discrepancies were faced, which can be good and healthy to debate openly about solutions and find the best one, but differences can likewise cause discomfort. Especially when decisions are made without involving the community or using their market dominance to steer the movement against planned and defined goals.
Solid being so lively with a community full of opinions can lead to a multitude of contrasting Server implementation even though it is described by technical reports and backed by a test-suite offering automatic auditing of implementations. Incidentally, harming the open-source idea for the same reason as mentioned in section \ref{challenges:lack}, where open-source developers fear to drain their time by developing an idea, which gets then replaced or made redundant by a more resourceful entity.
